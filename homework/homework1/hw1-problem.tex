\documentclass[12pt]{article}
\usepackage{times}

\topmargin 0.0cm
\oddsidemargin 0.2cm
\textwidth 16cm 
\textheight 21cm
\footskip 1.0cm


\usepackage{amsmath}
\usepackage{amsthm}
\usepackage{amsfonts}

\usepackage{pgf}
\usepackage{tikz}
\usetikzlibrary{arrows,automata}

\newcommand{\Sc}{{\mathcal S}}

\newtheorem{claim}{Claim}

\title{Homework problem}
\author{Volodymyr Kuleshov}
\date{}
%\date{June 6, 2012} % delete this line to display the current date

%%% BEGIN DOCUMENT
\begin{document}

%\baselineskip24pt

%\maketitle

\maketitle

\section{Question 1}

Suppose you have a probability distribution $P$ over random variables $X_1,X_2,...,X_n$ which all take values in the set $\Sc = \{v_1,...,v_m\}$, where the $v_j$ are some distinct values (e.g. integers, letters).

Suppose that $P$ satisfies the {\em Markov assumption}: for all $i \geq 2$ we have
$$ P(x_i | x_{i-1}, ..., x_1) = P(x_i | x_{i-1}). $$
In other words, $P$ factorizes as
$$ P(x_1,x_2,...,x_n) = P(x_1) P(x_1|x_2) \cdots P(x_n|x_{n-1}). $$
For each factor $P(x_i | x_{i-1})$ for $i\geq2$ you are given the probability $P(x_i = u | x_{i-1} = v)$ for each $u, v \in \Sc$ in the form of a $m \times m$ table. You are also given $P(x_1=v)$ for each $v$.

\begin{itemize}
\item Give an $O(m^2n)$ algorithm for solving the problem
$$ \max_{x_1,x_2,...,x_n} P(x_1,x_2,...,x_n). $$
Hint: think dynamic programming!
\item (Bonus) Give an $O(m^2n)$ algorithm that computes the marginal probability
$$ P(x_i) = \sum_{x_1,..., x_{i-1}, x_{i+1},...,x_n} P(x_1,..., x_{i-1}, x_i, x_{i+1},...,x_n) $$
for any $i$.
Hint: This algorithm will be very similar to your previous method, but you will have to do twice more work.
\end{itemize}

\end{document}
